\Startbonus

\hypertarget{science}{%
\chapter{Science}\label{science}}

\emph{John Snow Center for Medicine and Tower School of Doubt\\
(The Tower)}\\
\emph{October 3rd, 1998}\\

``Mr.~Abercrombie, Ms.~Ryan. How can I help you?'' asked the dean,
glancing at his wristwatch. ``It must be important if you've come to see
me during my office hours this week.''

Visiting the dean was relatively simple, but annoyingly tedious: you
simply pinned a note to the front of your robes about office hours, then
snapped a Safety Stick. Few students ever bothered, especially
considering how intimidating the former prodigy and current magical
titan could be. His inaugural speech to the Science Program students
hadn't been especially impressive -- a great deal of fuss about a ``pale
blue dot'' -- but some of the new students in the Program had felt faint
just from being in the Tower and in such proximity to the great man.
Craig Abercrombie and Siobhan Ryan thought this visit was necessary,
however.

As usual, every team in their year of the Science Program had been given
their project on Sunday. In this instance, each trio of students was
handed a small brown box containing the broken shards of a vase and a
small card of information. Craig, Siobhan, and Perry Paderau got a box
full of white-glazed pieces decorated with delicate designs in blue and
green. The card had informed them that this was formerly an Art Nouveau
vase created by Leon Solon, and told them that they were required to
``repair the vase'' without magic. \emph{You may use magic in any way
you please during the process, as long as no spell directly touches or
affects the pieces of the vase. Points will be awarded based on the
completeness of the restoration, overall aesthetic effect, and
creativity.}

``Well, sir, it's just got to be Muggle glue, right?'' said Craig.
``Nothing else you can do. Not much of a challenge. We were wondering if
you might talk to Professor Syracuse about it, and get him to change it
a bit.''

``I suggested this assignment, actually,'' said Dean of the Science
Program Harry Potter-Evans-Verres. He leaned back in his chair behind
the huge wooden table, adjusting his glasses, and gestured at a pile of
books at one end of the table. Craig recognized some of the textbooks
from the science program and several books on pottery styles and
history, along with a handful of note-filled parchments.

There was a brief pause as the two students absorbed this information,
then Siobhan spoke up. ``Sir, I'm not sure it fits with some of the
other projects we've done. They all needed\ldots{} well, you had to
think about them. This will just be\ldots{} tedious. Gluing things
together.''

``Don't underestimate the value of patience, Ms.~Ryan,'' said the dean.
``Having the fortitude to do something annoying and fiddly is a key
aspect of good science.'' He pushed himself back from the big table, and
stood up, gesturing vaguely. ``A few rooms away is a project I've been
working on for~\emph{years}, trying minor variations on the same thing
over and over again to try to find the exact shielding that will work
for my purposes. And I'll probably keep working at it tomorrow, and next
week, and so on. If you've decided on a way to complete your project,
don't quit just because it seems tedious. Most worthwhile things are
tedious at some point, so you should get used to tedium\ldots{} as long
as it's for a good purpose, and not just busywork.''

``This is just different than Professor Syracuse's previous assignments,
that's all,'' said Siobhan.

Craig nodded in agreement, and then his face lit up. ``There was
something about this sort of thing in one of our books\ldots{}'' He
walked over to the pile of books and notes that the dean had indicated.
He leafed through them until he found what he was looking for: a copy
of~\emph{Surely You're Joking, Mr.~Feynman!} Craig opened it and began
flipping through it, rapidly.

Some of the previous weekly projects from the Professor of Engineering
had been:

\begin{itemize}
\tightlist
\item
  \emph{Construct a way to view a basilisk with sufficient clarity that
  it could be effectively fought. Any means allowed, Muggle or magical.}
  Entries included glasses with mirrors built into them, blindfolds
  enchanted with~\emph{vitalis revelio}, a purchased pair of Muggle
  night-vision goggles, and a simple piece of parchment inscribed with
  the words, ``Use the Killing Curse and then view it as much as you
  want.''
\item
  \emph{Build upon last week's work studying Muggle agriculture, and
  suggest a new way to improve it in a well-structured essay. No minimum
  number of inches.} Answers were almost universally centered around
  either the use of magical creatures (interbreeding, pest control,
  etc.) or the production of fresh water (wide-scale weather management,
  enchanted saltwater filters, etc.) The most successful team pointed
  out that simply using Vanishing Rooms would result in the biggest
  improvement to Muggle agriculture, eliminating all the problems of
  preservation and transportation.
\item
  \emph{Go to the northeast corridor, take the second stairwell, go left
  down the hall, and enter the eighteenth room on the left. Once the
  door locks behind you, your team will have one hour to escape. You may
  not use your wands. You may bring anything else with you that you
  wish.} Students brought lockpicks enchanted with flawless function,
  battering rams transfigured to a small size, bottles of magical fire
  or Bundimun acid, and other things. Most plans had needed to be
  altered somewhat after the door vanished.
\item
  \emph{This is a Muggle device known as a ``mousetrap,'' used in place
  of the Vermexous Charm. It is missing the spring which would normally
  power it with mechanical energy. Make it work. Points will be awarded
  based on the effectiveness of the trap on a living mouse and
  creativity.} Most teams succeeded to get the trap to work, replacing
  the spring with twisted rope or other solutions. The two winning
  teams, however, found more innovative approaches. One team had put a
  lump of poisoned bait on the trap and ignored the device's original
  purpose. The other had tied the broken mousetrap to the back of a
  hungry kneazle.
\item
  \emph{Write an essay in three parts: (1) Where is an example of the
  Pareto Principle at work within Hogwarts? (2) Where can you find an
  example of the normal distribution in Hogwarts? (3) Identify a place
  where you would normally expect to find an example of either concept,
  even though it is not present. No minimum number of inches.}
\item
  \emph{Golden Snitches have been immobilized and hidden throughout the
  fifth floor. Find any Snitch, but remember that most sensory spells
  will not be effective. Do not go past the mungbeans or you will
  certainly become lost.} Only two teams had won. The first had gone and
  purchased a new Golden Snitch in Hogsmeade, pointing out that the
  rules didn't state~which~Snitch they needed to find. The other had
  researched the history of Quidditch's most famous cheaters and found a
  little-known fifteenth-century charm to divine the location of a
  Snitch. It used a distinctive wand motion. The following month, the
  Seeker for the Slough Sizzlers was fined a hundred Galleons and barred
  from competition.
\end{itemize}

After a moment of searching through the book, Craig had found the part
he wanted.

``Sir, remember when Mr.~Feynman goes to Brasilia and talks to them
about what they do with their science education?'' Dean Potter nodded;
it was one of the more famous parts of the book. ``Well, sir,
Mr.~Feynman says they have to choose a way because of `a good reason, a
sensible reason; not just because other countries do.'~'' The student
tapped the spot in the book.

``Yes, Mr.~Abercrombie. But I assure you, we're not doing this project
just because other engineering classes do it this way.'' The dean smiled
indulgently, and the expression paradoxically made him look very young.
He was only a few years older than them, after all.

``Yes, sir, but maybe you're assigning this project because you're doing
the sort of thing you think that Mr.~Feynman would do?'' said Craig,
questioningly. He closed the book and set it back down with the rest.

Siobhan frowned, shaking her head. ``Well, I don't know if that's it,
Craig. I just thought\ldots{}''

``It's a good point,'' said the dean, looking thoughtful. ``When I was
younger, I spent quite a bit of time feeling frustrated with my
teachers, and wishing I had a truly talented and creative tutor. I
wasn't quite prepared when I got my wish.'' He fell quiet for a moment,
and the students waited, a bit impatient despite their awe. The dean was
either referring to Albus Dumbledore or David Monroe, and it was a
dramatic reminder of how close they were to history\ldots{} but they
still wanted to leave as soon as possible.

``I'll think about it,'' said the dean. ``And before I give any more
suggestions to Professor Syracuse, I'll write out some clear lesson
objectives. Cleverness isn't a substitute for pedagogy, I suppose.''

``Thank you, sir,'' said Craig and Siobhan, just slightly out of unison.
They seemed discomfited by the end of the conversation; Craig was
tugging at his robes nervously and Siobhan was visibly sweating. They
left without another word.

\mybreak

The ensuing week was relatively normal -- or what passed for normal in
Hogwarts School of Witchcraft and Wizardry's Science Program, which was
not known for its normality. The lower-form students (in their first two
years of the Program) scurried in small packs from one class to another,
learning the rudiments of seven core subjects and one elective. The
upper-form students spent their time with fewer professors, studying the
rudiments of a few branches of science and doing labs. It was a
ruthlessly intense program, and more than half of the students quit
during their first year.

Professor Syracuse's afternoon class on Tuesdays, Wednesdays, and
Saturdays was a group of fourth-years. They had a swagger about them:
they'd survived three years of a course of study that was already
legendary for its difficulty, surpassing even the Salem Witches'
Institute's ``Trial by Fire'' graduate school of languages. In another
year, they'd be choosing independent courses of study in magical science
in the School of Doubt, working with Tower or Unspeakable researchers --
or even just beginning careers, if they wanted. They would be the third
graduating class of the Science Program, and they were on top of the
world.

Truth be told, the swagger in these fourth-years might explain why
Professor Santo Syracuse agreed so readily to the vase project when it
was suggested by Dean Potter-Evans-Verres. Such an assignment had good
prospects for teaching some arrogant teenagers a little humility.

``Sit down, sit down,'' snapped Professor Syracuse. ``Paderau! You heard
me! Sit down and be quiet! We have no time for your nonsense -- the
ladies aren't impressed. If you want to impress them, learn your
equations.''

The boy in question stood up from where he'd been crouched between two
witches and walked around their station back towards his own in the
back, wearing an expression of aggrieved innocence. He sat down between
Siobhan and Craig, making as much noise as possible as he settled his
elbows on the high table and his rear on the stool. His partners
exchanged a look of annoyance behind his back.

Professor Syracuse watched him intently for a moment to be sure that the
admonishment had been effective, then brightened as he turned to the
class as a whole. He was a thin man of average height, and gloriously
bald, with a shiny pink scalp and a mouth that twitched from side to
side when he was excited. He was often excited.

``Today we'll spend the first hour on project presentation, and then
after the break we'll be doing more work on friction,'' the professor
said, rubbing his hands together in anticipation and illustration.
``We'll try to hammer at least a few basic principles into you, so that
you're only woefully ignorant, and not completely ignorant. It will be a
rich, full day.'' He waggled his eyebrows in anticipation. ``Okay! Get
out your projects -- whatever you have, get it out, even if it's just
your notes! You can put your binders away for now. \emph{Do not}~spill
your flobberworm mucus or murtlap essence, or you will be
cleaning~\emph{everyone's}~station at the end of the afternoon.''

There was some shuffling and murmuring as people got themselves sorted,
taking out whatever their team had managed to complete that week. All
six of the teams appeared to have put together something in order to
repair the vase, but as everyone looked around, they saw a variety of
solutions.

``What did we get done, guys?'' Perry Paderau asked the other two, in a
hushed voice.

`` `We' didn't get anything done\ldots~\emph{Craig and I}~did finish
something, though,'' answered Siobhan, annoyed. She was arranging a
closed box in front of herself, carefully.

``Don't be that way, Ryan\ldots{} it's been crazy this past week,'' said
Perry, frowning. ``My dad wants me to come work for him when we get done
this year, and so I've been trying to get some extra help from Professor
Sprout in the evenings.'' Perry's father grew Sopophorous beans for
export.

``You didn't do anything, you just let Siobhan and I do it, and now
you're going to take credit,'' said Craig, irritably.

Perry turned to him, and spoke in a harsh whisper, ``Hey, you're not the
one who's expected to spend the rest of his life with baskets of
Mooncalf dung and a pair of silver scissors, okay? Do you know
how~\emph{often}~you need to sharpen silver scissors?'' He scowled. ``I
did all the work to get us out of that room last month, when the door
vanished, so have some mercy, will you?''

``This is the only time,'' said Siobhan.

``Fine!'' said Perry, a bit too loudly.

``Quiet over there!'' said Professor Syracuse, darting his gaze at their
team. He frowned. ``Again, Paderau? One point from Ravenclaw!'' Perry
groaned and slumped forward on the table. ``Okay, first team\ldots{}
Jess, Raphael, Sally\ldots{} what do you have?''

Two boys and a girl rose from their stools and walked awkwardly to the
front table. They set a vase down, carefully, as well as two small
bowls. The vase was small, brown, and extremely plain.

``Our solution was simple. We had a broken vase, and we needed to make a
working vase -- to `repair' it. So it seemed to us like the best thing
would be to just make a new vase, rather than trying to remake the old
one.'' She gestured at the table, and one of her teammates dipped his
fingers into one of the small bowls, lifting out a palmful of brown
powder. ``We took the pieces of the original vase and ground them down
into dust. Then we took that dust,'' she gestured again, and another
teammate displayed a handful of dark clay, ``and we added water, turning
it back into clay. We didn't use any magic on the pieces, before or
after we ground them down. We didn't even use~\emph{Aguamenti}~to create
the water -- we just used the tap.'' She sounded very proud.

``Then,'' she said, gesturing at the brown pot, ``we made a pot, and
asked a house elf to put it in the kilns for us the next time they fired
something. We got it back this morning, and here is the pot: clean and
new, and in one piece.''

The professor approached the front table, frowning. ``Full marks for
creativity, and I suppose this is a `complete restoration.'~'' He picked
up their pot, and examined it. ``I am actually surprised that this
worked. I wouldn't have thought that you'd be able to grind it down and
then just re-fire it. The vitrification\ldots{} hmm\ldots{}''

Professor Syracuse drew his wand and tapped the side of the pot twice,
saying, ``\emph{Aparecium}.'' The pot and the bowl of clay changed color
-- very slightly, tinting itself just a bit pink. The bowl of powder, on
the other hand, turned red. The professor turned to regard the trio of
students, eyebrow raised. ``Oddly, very little of the invisible dye
seems to have found its way into your new pot\ldots{} almost as though
you just mixed a little in with new clay, after discovering that your
plan wouldn't work.''

They muttered some excuses, but the professor was already waving them
back to their seats. ``If you want to remedy your low score today, then
I'd suggest you each write me thirteen inches on why you think your plan
didn't work, and what you should have done instead. I'd also suggest
availing yourself of the library, this time around. If you'd done even a
bit of research -- or if you'd been paying attention when we discussed
ceramics -- you'd have known about why this wouldn't work.''

Professor Syracuse turned back to the class. ``Next.''

The next two teams had simply glued the vase back together. One of the
teams had done much better than the other, and had clearly taken the
time to choose a specific kind of glue and practice, while the other
team's vase had small chips missing and beaded lines of overflow dried
along the seams. It even leaned a bit to the side.

Professor Syracuse commented on patience and conscientiousness as each
team presented their work. The team that would go last watched in
dismay, since it was obvious to everyone in the room that they had done
the worst job -- their glue didn't even look dry. One of them muttered a
charm under their breath, and tried to subtly position their box so that
it hid her efforts to use the warming spell on her work.

``Next.''

The fourth team had tried hard for the ``creative'' and ``aesthetics''
points as a strategy, and had used the pieces of their broken vase as a
mosaic on the outside of a different vase, breaking them into even
smaller fragments and arranging them in an attractive pattern. They held
up drawings they'd copied from a book with a Quarto Quickening Quill
from Queevel's, showing different examples of mosaics in art around the
world, as well as a large diagram indicating the best way to fit the
pieces and stick them in place. They were a very thorough group, and the
class was just lucky that they hadn't had time to make a diorama of a
Pompeiian antechamber. They looked to be leading the class this week,
easily.

``Next,'' said the professor, gesturing at Craig, Siobhan, and Perry.

The three of them got up. Siobhan carried the box with their project in
it. She set it down, stood in front of it, and took out the vase. The
white vase stood tall, and patterns of blue meshed with patterns of
green on its surface. All of the pieces had been placed neatly where
they belonged, but despite this care, the seams were clearly visible.
Indeed, they gleamed with gold. Thick lines of the metal traced the
joints between each piece. It was ostentatious, calling attention to the
damage rather than trying to hide it.

Perry looked horrified. ``This looks like we went mad,'' he hissed to
Siobhan.

``\emph{Shut up},'' she whispered back, fiercely.

``We wanted to do a technique from Japan called `kintsugi.' It's a
traditional Japanese craft, and part of an approach that doesn't try to
hide the history of a piece of broken ceramic, but instead make that
history part of the visible story of the piece,'' Craig said, sounding a
bit wooden and rehearsed. ``We couldn't find a shop that sold the sort
of lacquer that would work, which comes from a special tree, so we
experimented with different things -- potions and some goop from a Doxy
nest and that sort of thing that we thought might work.''

``This is Skele-Grow, reduced by half,'' said Siobhan, and she carefully
lifted the pot and held it up. ``We added a tiny bit of bone to activate
it, and dusted it with some powdered gold. Not a lot, and it turns out
to be cheaper than you'd think --''

``Because it's very ductile, so it can be made extremely thin,''
interrupted Perry, smiling as he was won over.

``\ldots and so our receipts still only total up to about five
Sickles,'' finished Siobhan, after an annoyed glance at Perry.

``Wonderful!'' exclaimed Professor Syracuse, looking positively
delighted. ``It looks beautiful -- and it shows not just creativity, but
real scholarship. This is actually -- my goodness -- this is actually
something specifically mentioned to me by the dean when we discussed
this project! He is quite a Japanophile, in fact, and we discussed
the~\emph{wabi-sabi}~aesthetic in particular!'' The professor shook his
head, marveling. ``I know we don't have~\emph{any}~books on the
topic\ldots{} how exactly did you learn about this technique?''

``Ah, well,'' said Craig, thinking about the notes on the table in the
Tower that he'd read while looking for the Feynman book. ``We remembered
what you said about `social engineering'\ldots{} it's easier if you
start with half the solution. So we asked around.''

The top sheet had read:\\

\emph{Santo, one final thought on my suggested assignment for next
week:}

\emph{I don't want to step on your toes, or make you feel like you have
to give this. We promised you broad discretion when Minerva first came
to you about your position in Killarney, and that hasn't changed. This
is just an idea I thought would be fun. The idea here isn't just to make
it difficult or tedious, since students will encounter enough of that
without our help. But we're giving them only the rudiments of a
scientific education here\ldots{} I want to challenge them as much as
possible. I mentioned kintsugi to you as one possible solution to the
project, but it's also a metaphor for the wizarding world. You're a
Muggleborn, and you were ostracized for relying on Muggle science for
your research on mermaids and evolution, so you know what we're up
against as we try to change society. These students are golden, but we
have to make them strong\ldots{} so they can hold together a broken
world.}

\qsource{H}

\mybreak

\emph{I think it goes back to my high school days. In computer class,
the first assignment was to write a program to print the first 100
Fibonacci numbers. Instead, I wrote a program that would steal passwords
of students. My teacher gave me an A.}

\qsource{Kevin Mitnick}

\Stopbonus
