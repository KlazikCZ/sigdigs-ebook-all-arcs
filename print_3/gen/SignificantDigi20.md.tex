\Startbonus

\hypertarget{analysis-and-thanks}{%
\chapter{Analysis and Thanks}\label{analysis-and-thanks}}

I really enjoyed~\emph{Harry Potter and the Methods of Rationality.
HPMOR}~excelled in its characterization, its intricate plot, its careful
phrasing and riddles, and in its use of dramatic tension and catharsis.
I loved the way it took aspects of the original~\emph{Harry Potter}~and
extrapolated them out into a world and timeline, using reasonably
pessimistic expectations to establish a small set of premises and then
draw the logical conclusion.

Some of the scenes I found particularly affecting were the following:

\begin{itemize}
\tightlist
\item
  Chapter six, where Harry talks about a childhood trauma when he felt
  unsafe, and we can feel that the author has shown us something very
  real and raw to him;
\item
  Chapter sixteen, when Harry has his first Battle Magic class and
  virtually the whole of the story is set in motion in a compact and
  subtle way;
\item
  Chapter forty-five, when the first Patronus 2.0 is cast and we read
  Harry's mental~\emph{cri de cœur}; and
\item
  Chapter eighty-one, the courtroom scene in which we learn everything
  we ever need to know about the awesome majesty of Minerva McGonagall.
\end{itemize}

When I set out to write~\emph{Significant Digits}, I tried to honor
everything I enjoyed and admired about~\emph{HPMOR.} The result is bound
to be unsatisfactory for some people, because not everyone was
fascinated by those same elements. Further, I was very specifically not
trying to mimic the original story. To imitate another author's voice
and recreate their patterns over an entire work would be very difficult
and not very fun, and I had no taste for the attempt.

I wanted to write a story about a changing world --
the~\emph{whole}~world -- as all the ambitions of the characters played
out and met their difficulties. I wanted to write a story about the
realization of the rationalism and humanism to which Harry aspired. I
wanted to write a story about extravagance: extravagant planning with
layered redundancies, extravagant characters whose passion led them to
discard the literal and logical conclusions of their own beliefs in
favor of still-greater pursuits, and extravagant events befitting the
process of optimizing the world.

I wanted~\emph{Significant Digits}~to answer some of the questions that
had lingered with me. These were big questions, and even in three
hundred thousand words, I couldn't completely answer all of them -- but
I did answer some. What was it like in the larger world of~\emph{Harry
Potter}, outside the confines of the school? How would magic and magical
races have shaped history and the hidden events behind them (ignoring
the well-meaning but utterly insane history of canon)? How could the
continued existence of this world be explained, given the elements we
knew to be present?

Lastly, of course, I sought to tell a story with interesting characters
and events that follow a rationally-unfolding plot, both at simple
levels and in intricate mental leaps. There were many twists that
everyone solved, some that only a few grasped, and a few that no one at
all predicted. This has been an amazingly intelligent and creative
group, and it was a considerable challenge to find the right balance.
Congratulations are due to those individuals who guessed some of the
biggest twists and puzzles, most particularly Reddit user /u/psinig, who
identified the Second of the Three.

In some respects, I have succeeded. In others, I have failed. I was
certainly overly ambitious, and should have given myself twice as long
and twice as many words. These limitations cramped plot development,
curtailed events, and required me to rely on implications in some
regards. But I do believe that I accomplished much of what I wanted to
create, and that I have done one more thing besides: left room for
others. There are other stories to be told. I'll write some of them, but
others have begun their own:~\href{http://www.2pih.com/}{*Orders of
Magnitude}*~is a prequel that's already begun.

There's a whole big world to play in.

There's a lot I would change now, even though I'm pretty happy with the
story. It's my first work of this length, and my first work of serial
fiction, and naturally there are all kinds of changes I would make in
hindsight. I became a better writer over the course of this past year,
and a more critical thinker. I should probably have cut back on some of
the secondary storylines, in retrospect, since I didn't have time or
room to do them justice.

There is one chapter, though, that I would not change and that I am
utterly happy with -- a chapter in which I did every little thing I
wanted to do, and yet somehow arrived at something that was even more
than the sum of all those parts.
\href{http://www.anarchyishyperbole.com/2015/07/significant-digits-chapter-fourteen.html}{Chapter
Fourteen, Azkaban}, is everything for which I have aimed, and will
continue to aim in my fiction. I can recommend that chapter to you, at
the least, with a full and proud and happy heart.

As for the rest, that's for you to decide.

Gratitude is due to many people.

Writing the story would have been quite literally impossible if it
weren't for the extraordinary efforts of 4t0m, go\_on\_without\_me,
pa55word, and a final editor who wishes to go unnamed. Their tireless
willingness to sweat the small stuff despite unreasonably short
deadlines, challenge poor phrasing or poor ideas, and cheer on our joint
successes was extraordinary. This was their story and their
accomplishment, too. Thank you all.

Readers and commenters have provided an enormous amount of support and
constructive criticism, both of which have helped me improve as the
story continued. I have been writing for a long time, but this is the
first thing that's ever gotten this kind of response, and a large part
of that was that the community of~\emph{HPMOR}~fans is so creative and
clever and kind. Amazing individuals improved my website, fixed up the
subreddit, donated a laptop when I complained about a green tint on my
screen (!), and put together PDF and EPUB versions of the text. Thank
you all.

Generous patrons on Patreon provided a real reason to keep going when
things were hard. While I frequently remind people to consider their
priorities before donating to a writer, it's also true that money is the
unit of caring. Patronage provided a message of support and very real
assistance that could not be explained away as courtesy or indifferent
politeness. Thank you all.

Eliezer Yudkowksy wrote something genuinely new and good, and inspired
legions. And I certainly wouldn't have begun the story at all if it
hadn't been for his gentle encouragement and reception when I first
posted a snapshot of my ideas. Thank you.

Nothing would have been possible, or worthwhile, if it hadn't been for
my wife Lizzie. She walked with me in the woods while I talked about
ideas. She proofread all the early chapters. She took the cover picture.
I know that there is some ineluctable grace in this world, because I
know her. Thank you.

My next story will be
the~\href{http://www.anarchyishyperbole.com/p/conquest.html}{\emph{The}~\emph{Consolation
of Conquest}}. It will begin in about a month, and updates will come at
a more reasonable fortnightly basis. Please subscribe to
my~\href{http://eepurl.com/bLNHdz}{mailing
list}~or~\href{http://www.anarchyishyperbole.com/feeds/posts/default}{RSS
feed}~or~\href{https://www.reddit.com/r/AiH}{subreddit}~to receive
updates.

Thank you.

\Stopbonus
